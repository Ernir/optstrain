\documentclass[12pt]{article}
\usepackage[top=2.5cm, bottom=2.5cm, left=2.5cm, right=2.5cm]{geometry}
\usepackage[utf8]{inputenc}
\usepackage[icelandic]{babel}
\usepackage[T1]{fontenc}
\usepackage[sc]{mathpazo}
\usepackage[parfill]{parskip}
\usepackage{booktabs}
\usepackage{amsmath}
\usepackage{color}
\usepackage{graphicx}
\usepackage{wrapfig}
\usepackage[pdftex,bookmarks=true,colorlinks=true,linkcolor=blue,urlcolor=blue]{hyperref}

\title{Framleiðsla á humulene í Saccharomyces cerevisiae}
\author{Eiríkur Ernir Þorsteinsson \and Jónas Tryggvi Stefánsson}

\begin{document}

\maketitle

\begin{abstract}
Útdráttur
\end{abstract}

% Myndir verða mikilvægar, fá 2-4
% Byrja á myndunum!
% Hugmynd: Flæðirit yfir virkni reikniritsins

\begin{figure}
\caption[OptStrain reikniritið]{Skref OptStrain reikniritsins sem lýst er í \ref{sec:optstrain}.}
\label{fig:flaedirit}
\includegraphics[width=\textwidth]{Pics/OptStrainOverview}
\end{figure}


\section{Inngangur}
Vísun í OptStrain-greinina svo þetta compile-i \cite{pharkya2004optstrain}

\section{Aðferð}
% Fyrsti textinn sem er skrifaður
% Byrja á undirtitlum
% Mikilvægt að strúktúra málsgreinarnar vel
% Mikilvægt að tengja 
\subsection{OptStrain}
\label{sec:optstrain}
\subsubsection{Almenn lýsing}
\subsubsection{Útfærsla}

\section{Niðurstöður}

% Undirtitlar hér líka

\section{Samantekt}
% Byrjar oftast eins: Here we set out to...

% \paragraph{Samantekt á niðurstöðum}
% \paragraph{Hvernig falla niðurstöðurnar að öðrum þekktum niðurstöðum}
% Ræða hvern undirtitil í niðurstöðu-hlutanum sérstaklega
% Aftur samantekt á síðustu málsgreinum, summary of impact

\appendix
%%%%%%%%%%%%%%
% HEIMILDASKRÁ
%%%%%%%%%%%%%%
\bibliographystyle{plain}
\bibliography{bioinfo.bib}

\end{document}