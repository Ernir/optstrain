\documentclass[11pt]{article}
\usepackage[top=2.5cm, bottom=2.5cm, left=2.5cm, right=2.5cm]{geometry}
\usepackage[utf8]{inputenc}
\usepackage[icelandic]{babel}
\usepackage[T1]{fontenc}
\usepackage[sc]{mathpazo}
\usepackage[parfill]{parskip}
\usepackage{booktabs}
\usepackage{amsmath}
\usepackage{color}
\usepackage{graphicx}
\usepackage{wrapfig}
\usepackage{minted}
\usepackage[pdftex,bookmarks=true,colorlinks=true,linkcolor=blue,urlcolor=blue]{hyperref}

\title{OptStrain}
\author{Eiríkur Ernir Þorsteinsson \and Jónas Tryggvi Stefánsson}
\date{8. apríl 2016}
%\setcounter{secnumdepth}{0}

\begin{document}

\maketitle

\section{Lýsing á reikniritinu}

Um er að ræða útfærslu á reikniritinu OptStrain\cite{pharkya2004optstrain} sem notað er til að finna uppskriftir að örverum sem henta betur til framleiðslu á gefnu efni. OptStrain er framkvæmt í fjórum skrefum:

Reikniritið gerir ráð fyrir grunnmódeli af lífveru sem hentar til línulegrar bestunar.
\begin{enumerate}
 \item Efnahvörfum úr stórum efnahvarfagagnagrunni er bætt við grunnmódel af gefinni lífveru til að fjölga mögulegum pathway-um
 \item Hámörkun á seytingu æskilegs efnis við gefnar aðstæður er sett upp sem línulegt bestunarverkefni og leyst
 \item Lágmörkun á fjölda viðbættra efnahvarfa (úr 1. skrefi) er sett upp sem MILP-verkefni og leyst
 \item OptKnock-reiknirit\cite{burgard2003optknock} er keyrt á breytta módelið til að reyna að finna útslátt sem kúplar seytingu efnisins við vöxt
\end{enumerate}

\section{Um útfærslu}
Til viðbótar við almenna útfærslu sem lýst er að ofan bætist við upphafsskref þar sem efnahvörf sem nauðsynleg eru til framleiðslu á efninu sem framleiða skal, humulene, er bætt við \href{http://yeast.sourceforge.net/}{Yeast 7 módelið} af \emph{Saccharomyces cerevisiae}. Þetta er gert fremst í \nameref{code:augmentModel}.

Útfærsla er í Matlab. Hún krefst uppsettra Cobra, CVX og Gurobi pakka. Módelið sem um ræðir er Cobra-módel.
\subsection{Útvíkkkun á módeli}
Útvíkkun á módeli fer fram í \nameref{code:augmentModel}. Hér er efnahvörfum hlaðið úr Excel-grunni. Mikið af vinnunni fer í að þýða þau efnahvarfaauðkenni sem gefin eru í grunninum yfir á form sem hentar módelinu. Þegar auðkenni finnast ekki í módelinu eru þau búin til. Öllum efnahvörfum í grunninum (sem reyndust á skiljanlegu sniði) er bætt við grunnmódelið. Þessi hluti reikniritsins gekk að óskum.
\subsection{Bestun á humulene-framleiðslu}
Humulene-framleiðsla við loftfirrtar aðstæður er bestuð í \nameref{code:calcBaseYield} með nú kunnuglegum aðferðum. Þessi hluti reikniritsins er einnig skv. lýsingu.
\subsection{Lágmörkun á fjölda viðbættra efnahvarfa}
\label{sec:lagmorkun}
CVX og Gurobi bestunarpakkarnir eru notaðir til að lágmarka fjölda viðbættra efnahvarfa. Sá kóði er staðsettur í \nameref{code:OptStrain} skránni sjálfri vegna scoping-reglna Matlab.

Hér er á þessu stigi að finna \emph{mikið vandamál}. Af enn óþekktum ástæðum tekur CVX-pakkinn ekki við gefnum líkönum af gersveppnum. Þetta á við um módel af grunnútgáfu sveppsins og viðbættar útgáfur. Þessi hluti virkar vel á einfalda \texttt{Ecoli\_core\_model} líkanið, en umfangsmiklar prófanir eru vegna þessarar takmörkunar.
\subsection{Útsláttur til vaxtarkúplunar}
Einföld útgáfa af OptKnock, \texttt{simpleOptKnock} úr Cobra , er notuð til að finna vænlega útslætti. Þessi hluti er takmarkaðri en lýsing OptStrain segir til um, en dugar til að fá nokkra mynd af einföldustu útsláttum.

Þessi hluti er einnig staðsettur í \nameref{code:OptStrain} skránni, neðst.
\section{Niðurstöður}
Útslættir hentugir til vaxtarkúplunar á humulene-framleiðslu hafa ekki fundist vegna takmarkana sem lýst er undir \nameref{sec:lagmorkun}.

Ein niðurstaða er þó bæði gremjuleg og áberandi - humulene-framleiðsla eykst ekki við aukningu á módelinu í skrefi 1, hún helst í 0.2222 þrátt fyrir viðbætur við staðalupptöku á glúkósa. Áhugavert væri að athuga hvort þetta á við um aðra kolefnisgjafa líka.
\appendix
\section{Forritskóði}
Allan forritskóða má nálgast á \url{https://github.com/Ernir/optstrain}, valinn hluti er sýndur hér að neðan.

Kóðinn er birtur í því ástandi sem hann er.
\subsection{OptStrain.m}
\label{code:OptStrain}
\inputminted[linenos]{matlab}{../OptStrain.m}
\subsection{augmentModel.m}
\label{code:augmentModel}
\inputminted[linenos]{matlab}{../augmentModel.m}
\subsection{calcBaseYield.m}
\label{code:calcBaseYield}
\inputminted[linenos]{matlab}{../calcBaseYield.m}
%%%%%%%%%%%%%%
% HEIMILDASKRÁ
%%%%%%%%%%%%%%
\bibliographystyle{plain}
\bibliography{bioinfo.bib}

\end{document}